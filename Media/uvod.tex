Metaheuristika se odnosi na oblikovni obrazac koji ugrađuje prethodno znanje u izgradnju raznih optimizacijskih algoritama. Smatra se potpodručje stohastičke optimizacije. Stohastička optimizacija je kombinacija algoritama i slučajnih procesa radi pronalaženja optimalnih rješenja. 

Prirodom nadahnute metaheuristike su one koje crpe ideje za pretraživanje prostora stanja iz prirode. Algoritmi rojeva su potkategorija prirodno nadahnutih metaheuristika koji su nadahnuti kolektivnim ponašanjem životinja ili kukaca. Roj se sastoji od više individua čije je ponašanje vrlo jednostavno, ali zajedno mogu djelovati inteligentno. 

Postoje mnogi takvi algoritmi, najpoznatiji su mravlji algoritam i algoritam pčela. U ovom seminaru razmotrit će se manje poznati algoritam majmuna i algoritam krijesnica.